\documentclass[journal]{IEEEtran}

\ifCLASSINFOpdf
  \usepackage[pdftex]{graphicx}

   \graphicspath{{../pdf/}{../jpeg/}}
 


\hyphenation{op-tical net-works semi-conduc-tor}


\begin{document}


\title{Parcial No.1}


\author{Juan~José~Loaiza~Lujan.\\Universidad~del~Valle~sede~Zarzal,~Valle~del~Cauca,~Colombia.\\juan.loaiza.lujan@correounivalle.edu.co\\Programa:Algoritmia y programación\\Profesor:Paul Angarita Jimenez
\thanks{M. Shell was with the Department
of Electrical and Computer Engineering, Georgia Institute of Technology, Atlanta,
GA, 30332 USA e-mail: (see http://www.michaelshell.org/contact.html).}
\thanks{J. Doe and J. Doe are with Anonymous University.}
\thanks{Manuscript received April 19, 2005; revised August 26, 2015.}}





\section{Parcial No.1}
\subsection{Juan~José~Loaiza~Lujan.\\Universidad~del~Valle~sede~Zarzal,~Valle~del~Cauca,~Colombia.\\juan.loaiza.lujan@correounivalle.edu.co\\Programa:Algoritmia y programación\\Profesor:Paul Angarita Jimenez}

\begin{figure}[h!]
\centering
\includegraphics[width=5.0cm]{C:/Users/USUARIO/Documents/AyP/Parcial 1.1/Identificadordenumeros.jpeg}
\caption{Identificador de números amigos}
\label{figure_sim}
\end{figure}


\begin{figure}[h!]
\centering
\includegraphics[width=5.0cm]{C:/Users/USUARIO/Documents/AyP/Parcial 1.1/Matrisinversa1.jpeg}
\caption{Primera parte de la Matris inversa en c++}
\label{figure_sim}
\end{figure}

\begin{figure}[h!]
\centering
\includegraphics[width=5.0cm]{C:/Users/USUARIO/Documents/AyP/Parcial 1.1/Matrisinversa1.jpeg}
\caption{Segunda parte de la matris inversa en c++}
\label{figure_sim}
\end{figure}

\begin{figure}[h!]
\centering
\includegraphics[width=5.0cm]{C:/Users/USUARIO/Documents/AyP/Parcial 1.1/Pruebamatris.jpeg}
\caption{Prueba de la Matris}
\label{figure_sim}
\end{figure}


\begin{figure}[h!]
\centering
\includegraphics[width=8.0cm]{C:/Users/USUARIO/Documents/AyP/Parcial 1.1/newton1.jpeg}
\caption{Primera parte del metodo de newton}
\label{figure_sim}
\end{figure}

\begin{figure}[h!]
\centering
\includegraphics[width=8.0cm]{C:/Users/USUARIO/Documents/AyP/Parcial 1.1/newton2.jpeg}
\caption{Segunda parte del metodo de newton}
\label{figure_sim}
\end{figure}


\begin{figure}[h!]
\centering
\includegraphics[width=8.0cm]{C:/Users/USUARIO/Documents/AyP/Parcial 1.1/newtonprueba.jpeg}
\caption{Prueba de newton}
\label{figure_sim}
\end{figure}


\end{document}